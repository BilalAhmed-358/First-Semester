\documentclass[12pt]{article}
\usepackage[utf8]{inputenc}
\usepackage{ulem}
\usepackage{graphicx}
\usepackage{caption}
\usepackage{amsmath}
\title{Practice}
\author{Bilal Ahmed Khan}
\date{19th January 2021}

%\setlength{\parindent}{}
\begin{document}
\maketitle
\begin{abstract}
    This is an abstract. It contains \textbf{abstract information} about the text I have discussed in this document. 
    
    This document is basically made to keep a track record of things which I daily learn to do in latex :).
\end{abstract}
\textbf{Following is a motivational image.}
\begin{figure}[h]
    \centering
    \includegraphics[width=0.5\textwidth]{Desktop Background.png}
    \caption{Motivational Pic}
    \label{fig: motivational}
\end{figure}
%As it is shown in \ref(motivational)

Holy shit how beautiful is this picture! \ref{fig: motivational}
which is on page number \pageref{fig: motivational}

Following are the things I hate about myself:
\begin{itemize}
\item I am lazy
\item Somewhat over-confident also ;)
\item Have respect issues
\end{itemize}
\newpage
\noindent\textbf{This is an ordered list and these are the things I like about myself}
\begin{enumerate}
\item I am resilient
\item I am stubborn (Yes I count it as an advantage)
\item I try to speak truth
\end{enumerate}

Below is the renowned equation of \textit{Einstein}\[E=mc^2\]
I have also written it in \textbf{\large BOLD}, \textit{\large ITALIC} and \underline{\large UNDERLINE} to emphasize the matter!.

\textit{\textbf{\uline{Einstein introduced this equation to the world in His famous series of lectures given in Canterbury university}}}

You can also write equations in a paragraph like this
 \[2aS=vf^2-vi^2\]

Or you can write them inline like this:

Third equation of motion: \(2aS=v_f^2-v_i^2\)

All three equations of motions are written below:

\begin{align}
    v_f =v_i+at\\
    %S =v_i*t+1/2a*t^2\\
    %2aS =v_f^2-v_i^2 
\end{align}
\newpage
Below is my practice of paragraphs and sections
\section{This is a section}
\subsection{This is a subsection}
\subsubsection{This is a subsubsection}
\paragraph{This is a paragraph}
\subparagraph{This is a subparagraph}

\textbf{The heirarchy of Chapters and text is as follows:}
\begin{enumerate}
    \item Part
    \item Chapter
    \item Section
    \item Subsection
    \item Subsubsection
    \item Paragraph
    \item SubParagraph
\end{enumerate}
\textbf{P.S.} Part and chapter function only work in report and book classes and not in other so keep that in mind when using such options.

\noindent{\large Now I'll practice making tables in LATEX.\\
Following is my practice.}\\[\baselineskip]

\noindent\textbf{This is the list of me and my university friends:}
\begin{center}
{\large \begin{tabular}{||c|c|c||}
    \hline\hline
   Student ID & Student Name & Roll No.\\\hline\hline
    01 & Bilal Ahmed Khan & 0183\\\hline\hline
    02 & Mohammad Wamiq Akram & 1857\\\hline\hline
    03 & Zulnoor Siddiqui & 1090\\\hline\hline
\end{tabular}}
\end{center}
\end{document}
